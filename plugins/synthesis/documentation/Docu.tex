\documentclass[
a4paper, 
12pt,
]{article}

\usepackage{parskip}
\usepackage{amsmath}
\usepackage{listings} 

\begin{document}
    \section{Overview}
    This is the documentation of Changming Yang's implementation of his master thesis on the topic "Effective Control Synthesis under Partial observation" \\
    the whole work is based on the cpp library libFAUDES, and is also implemented in cpp

    \section{Problem Definition}
    According to the article, the goal is to design a complete, deadlock-free suprevisor V for plant G such that 
    
    \begin{equation}
        S(V/G)\subseteq E
    \end{equation}

    where the plant is designed to be a Rabin Automata with a trivial acceptance condition, which can be simplfied into a Buechi automata as follow:
    
    \begin{equation}
        \begin{split}
            G&=(\Sigma, X_G, \delta _G , x_{0G} ,(X_G,X_G))  \\
            G&=(\Sigma, X_G, \delta _G , x_{0G})
        \end{split}
    \end{equation}

    and the Specification is a Rabin Automata with a single acceptance condition:

    \begin{equation}
        E=(\Sigma, X_E, \delta _E , x_{0E} ,(R_E,I_E))
    \end{equation}

    \section{Operation}
    \subsection{Define RabinPairs}
    Because libfaudes has a template of Generator, I can construct a Rabin Automata by only defining a Rabin Acceptance Condition (one single rabin pair is utilised in our case), in the practical construction, multiple RabinPairs can be inserted. \\

    Constructors and Destructors:
\[
\texttt{class AttributeRabinPairs : public AttributeVoid } \\
\]

Default Constructor:
\[
    \texttt{AttributeRabinPairs()}
\]
Copy Constructor:
\[
    \texttt{AttributeRabinPairs(const AttributeRabinPairs\& rOther)}
\]
Destructor:
\[
    \texttt{virtual ~AttributeRabinPairs()}
\]
Assignment operator:
\[
    \texttt{AttributeRabinPairs\& operator=(const AttributeRabinPairs\& rOther)}
\]
Equality comparison operator:
\[
    \texttt{bool operator==(const AttributeRabinPairs\& rOther) const}
\]
Inequality comparison operator:
\[
    \texttt{bool operator!=(const AttributeRabinPairs\& rOther) const}
\]
    
where the Rabin Acceptance condition is defined with: \\
$Inf(\pi)\bigcap R \ \neq \emptyset$, There exists at least one element in R-set can be infinitely visited.\\
$Inf(\pi)\bigcap I=\emptyset$, All the elements in R-set are finitely visited.\\

Add a Rabin pair of R-set and I-set:
\[
    \texttt{void AddRabinPair(const StateSet\& rRSet, const StateSet\& rISet)}
\]
Set Rabin pairs attribute for a Generator:
\[
    \texttt{void SetRabinPairsAttribute(Generator\& rGen, const AttributeRabinPairs\& rAttr)}
\]
Get Rabin pairs attribute from a Generator:
\[
    \texttt{AttributeRabinPairs GetRabinPairsAttribute(const Generator\& rGen)}
\]
Write Generator to console, including Rabin pairs:
\[
    \texttt{void CustomDWrite(const Generator\& rGen)}
\]


\subsection{Build Controlled System}
Before construct a Controlled System, a control pattern $\mathbf{C}$ must be designed by extending the alphabet $\varSigma $, then a Controlled System with a trivial Acceptance condition can be designed as below, furthermore, the Specification is also extended with the same alphabet.

\begin{equation}
    G_C=(\Sigma\times C, X_G, \delta _{GC} , x_{0G}) 
\end{equation}
\begin{equation}
    E_C=(\Sigma\times C, X_E, \delta _{EC} , x_{0E},(R_{EC},I_{EC})) 
\end{equation}



Apply control patterns to a Generator, preserving Rabin pairs:
$$
\begin{array}{l}
\verb|void ApplyControlPattern(const Generator& rPlantGen, const EventSet&|\\
\verb|rControllableEvents, Generator& rControlledGen)|\\ 
\verb|   @rPlantGen: The plant Generator  |\\
\verb|   @rControllableEvents: Set of controllable events  |\\
\verb|   @rControlledGen: Output controlled Generator  |\\
\end{array}
$$

\subsection{Compute $A_C$}

$A_C$ is the Product of controlled system and Specification $\mathbf{G_C}\times\mathbf{E_C}$

\begin{equation}
    A_C=(\Sigma\times C, X_G \times X_E, \delta _c , (x_{0G},x_{0E}) ,(R_c,I_c))
\end{equation}
one thing need to be take care is, the Rabin Pairs result of a universal Product of two Rabin Automata should be:
\begin{equation}
    \begin{split}
        R_c&=R_1 \times X_2 \cup  X_1 \times R_2 \\
        I_c&=I_1 \times X_2 \cup  X_1 \times I_2 \\
    \end{split}
\end{equation}
In our case, since there is not a $R_1$ and $I_1$, so the final rabin pair can be simplfied to:
\begin{equation}
    \begin{split}
        R_c&= X_1 \times R_2  =X_G \times R_E\\
        I_c&= X_1 \times I_2  =X_G \times I_E\\
    \end{split}
\end{equation}

To simplfy the code, I use the function $Product(Plant,Specification)$ from libFAUDES to compute the product, return a new system GC; Seperately compute product of Rabin pairs from two Generators, then insert the new RabinPairs into GC:

\[
Product(Plant,Specification,ompositionMap,GC)
\]

$$
\begin{array}{l}
\verb|void ProductRabinPair(const Generator& rPlant, const Generator& rSpec,|\\ 
\verb|const std::map<std::pair<Idx, Idx>, Idx>& rCompositionMap, Generator& GC)|\\
\verb|  @rPlant: First Generator|\\
\verb|  @rSpec: Second Generator|\\
\verb|  @rCompositionMap: Mapping from state pairs to product states|\\
\verb|  @GC: Output product Generator|\\
\end{array}
$$

\subsection{Compute the Projection without determinization $A_o$}

The natural Projection is used to simplfy the Observation Mask, $RabinProjectNonDet()$ is imitated based on the function $ProjectNonDet()$ from libFAUDES to compute the Projection \textbf{without} determinization, I will later use another algorithm to  determinize.\\
\begin{equation}
    A_o=(\Sigma\times C, X_G \times X_E, \delta _o , (x_{0G},x_{0E}) ,(R_c,I_c))
\end{equation}
$$
\begin{array}{l}
\verb|void RabinProjectNonDet(const Generator& rGen, const EventSet& |\\ 
\verb|rProjectAlphabet, rProjectAlphabet,Generator& rResGen)|\\
\verb|  @rGen: Input Generator|\\
\verb|  @rProjectAlphabet: Alphabet to project onto|\\
\verb|  @rResGen: Output projected Generator|\\
\end{array}
$$

\subsection{PseudoDeterminization}

TODO



















\end{document}